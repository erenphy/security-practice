\documentclass{labart}

\renewcommand{\abstractname}{概要}
\title{ROP 攻击实验}
\author{网络与系统安全综合实验}
\date{实验截止: XXXX.xx.xx 星期X 23:59}
\labtitle{lab1}

\begin{document}

\maketitle

% 本文件使用XeLaTeX编译生成,确保编译前已安装所需的Package

\begin{abstract}
本手册主要介绍ROP攻击实验的相关内容,并推荐常用工具及使用方法。
\end{abstract}


\section{实验相关}
\subsection{实验内容}
本次实验为网络与系统安全综合实验第一次实验(以下称为Lab1),主要内容是ROP攻击方法的利用。Lab1一共包括四个题目:
\begin{lstlisting}[numbers=none,frame=none]
- lab1-1:got hijack
- lab1-2:rop
- lab1-3:rop
- lab1-4:rop
\end{lstlisting}

\subsection{提交内容}
Lab1需要提交实验报告(PDF/Word)和每道题目的解题脚本,实验报告需要完整反映题目解答过程和最后的答案,实验报告和解题脚本一同压缩为.zip格式上交。

\begin{lstlisting}[numbers=none,frame=none]
命名格式:
压缩包:lab1-学号-姓名.zip                    e.g. lab1-17000000001-张三.zip
报告:  lab1-学号-姓名.[doc/pdf]              e.g. lab1-17000000001-张三.pdf
脚本:  lab1-[1-4].[c/py/...]       e.g. lab1-2-17000000001-张三.py
\end{lstlisting}


\section{工具相关}
\noindent 推荐一些常用的调试/ROP常用工具。

\vspace{0.5cm}

\noindent gdb插件 peda 或者 pwndbg

\begin{lstlisting}[numbers=none,frame=tb, language=bash]
# peda
git clone https://github.com/longld/peda.git ~/peda
echo "source ~/peda/peda.py" >> ~/.gdbinit
# pwndbg
git clone https://github.com/pwndbg/pwndbg
cd pwndbg
./setup.sh
\end{lstlisting}
\vspace{0.5cm}
\noindent pwntools
\begin{lstlisting}[numbers=none,frame=tb, language=bash]
pip install pwntools
\end{lstlisting}
\vspace{0.5cm}
\noindent ROPgadget and ropper
\begin{lstlisting}[numbers=none,frame=tb, language=bash]
pip install ropgadget
pip install ropper
\end{lstlisting}
\end{document}